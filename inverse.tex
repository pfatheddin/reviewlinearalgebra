\documentclass{ximera}
 \usepackage{amsmath}
\author{Parisa Fatheddin}
\title{Inverse of Matrices}
\begin{document}
\begin{abstract}
Methods on finding the inverse of matrices is recalled in this section. 
\end{abstract}
\maketitle
\begin{center}
\youtube{ZD6WIS4PqeY}
\end{center}

\begin{exercise}
\begin{hint}
For invertible matrices $A_{1}, A_{2},...,A_{k}$,
\begin{equation*}
\left(A_{1}A_{2}...A_{k}\right)^{-1} = A_{k}^{-1} A_{k-1}^{-1}...A_{2}^{-1} A_{1}^{-1}
\end{equation*}
\end{hint}
For
\[ A= \left(\begin{array}{cc}
  -1 & 3   \\
  0 &  5
\end{array}\right), \hspace{.5cm} B= \left(\begin{array}{cc}
  2 & -4   \\
  -1 &  0
\end{array}\right),
\]
a. Find $A^{-1}$ and $B^{-1}$.

\begin{prompt}
\[ A^{-1}= \left(\begin{array}{cc}
  \answer{-1} & \answer{3/5}   \\
  \answer{0} & \answer{1/5}
\end{array}\right), \hspace{.5cm} B^{-1}= \left(\begin{array}{cc}
  \answer{0} & \answer{-1}   \\
  \answer{-1/4} & \answer{-1/2}
\end{array}\right)\]
\end{prompt}

b. Find $(BA)^{-1}$.

\begin{prompt}
\[(BA)^{-1} = \left(\begin{array}{cc}
  \answer{-3/20} & \answer{7/10}   \\
  \answer{-1/20} & \answer{-1/10}
\end{array}\right)
\]
\end{prompt}

c. If
\[C = \left(\begin{array}{cc}
  0&2\\
  -3&0
\end{array}\right)
\]
what is $(CAB)^{-1}$?

\begin{prompt}
\[(CAB)^{-1}= \left(\begin{array}{cc}
  \answer{-1/10} & \answer{0}   \\
  \answer{-1/8} & \answer{-1/12}
\end{array}\right)
\]
\end{prompt}
\end{exercise}
\begin{exercise}
Consider the following system of equations.
\begin{eqnarray*}
3x+ 2y&=& -19\\
x-5y &=& 5
\end{eqnarray*}
a. Solve by Gaussian elimination method. \\
b. Solve by finding the inverse of the coefficient matrix.

\begin{prompt}
a.
\[\left(\begin{array}{cc|c}
\answer{3} & \answer{2} & \answer{-19}\\
\answer{1} & \answer{-5} & \answer{5}
\end{array}\right) \hspace{.2cm} \rightarrow \hspace{.2cm}
\left(\begin{array}{cc|c}
1 & \answer{-5} & 5\\
0 & \answer{17} & \answer{-34}
\end{array}\right)\hspace{.2cm} \rightarrow \hspace{.2cm}
\left(\begin{array}{cc|c}
1 & \answer{-5} & 5\\
0 & 1 & \answer{-2}
\end{array}\right)
\]
leading to
\begin{equation*}
x= \answer{-5}, \hspace{.5cm} y = \answer{-2}
\end{equation*}

b.
\[A= \left(\begin{array}{cc}
\answer{3} & \answer{2}\\
\answer{1} & \answer{-5}
\end{array} \right), \hspace{.5cm} A^{-1}= \left(\begin{array}{cc}
\answer{5/17} & \answer{2/17}\\
\answer{1/17} & \answer{-3/17}
\end{array} \right)
\]
Thus,
\[
\textbf{x} = \left(\begin{array}{c}
\answer{-95/17} + \answer{10/17}\\
\answer{-19/17} + \answer{-15/17}
\end{array}\right)
\]
giving,
\begin{equation*}
x= \answer{-5}, \hspace{.5cm} y = \answer{-2}
\end{equation*}

\end{prompt}
\end{exercise}

\begin{exercise}
  Consider the following system of equations,
  \begin{align*}
    2x-3y&= 10\\
    4z -2x &= 2\\
    y+2z &= -4
  \end{align*}

  \begin{enumerate}
  \item Solve using Gaussian elimination.
    \begin{prompt}
      The augmented matrix is
      \[
        \left(
          \begin{array}{ccc|c}
            \answer{2} & \answer{-3} & \answer{0} & \answer{10}  \\
            \answer{-2} & \answer{0} & \answer{4} & \answer{2}  \\
            \answer{0} & \answer{1} & \answer{2} & \answer{-4}
          \end{array}
        \right)
        \quad \rightarrow \quad
        \left(
          \begin{array}{ccc|c}
            1 & \answer{-3/2} & \answer{0} & 5  \\
            0 &1 & 2& -4  \\
            0 & 0 & 1 & \answer{0}
          \end{array}
        \right)
      \]
      giving $x = \answer{-1}, y= \answer{-4}, z= \answer{0}$.
      % \quad(if there is no solution write N/A).
    \end{prompt}

  \item Solve using the inverse of the coefficient matrix.
    \begin{prompt}
      \[
        (A \, | \, I) = \left(
          \begin{array}{ccc|ccc}
            \answer{2} & \answer{-3} & \answer{0} &  \answer{1} & \answer{0} & \answer{0}  \\
            \answer{-2} & \answer{0} & \answer{4} & \answer{0} & \answer{1} & \answer{0}\\
            \answer{0} & \answer{1} & \answer{2} &\answer{0} & \answer{0} & \answer{1}
          \end{array}
        \right)
      \]
            \[
        \quad \rightarrow \quad
        \left(
          \begin{array}{ccc|ccc}
            1 & 0 & 0 &  \answer{1/5} & \answer{-3/10} & \answer{3/5}  \\
            0 & 1 & 0 & \answer{-1/5} & \answer{-1/5} & \answer{2/5}\\
            0 & 0 &1 &\answer{1/10} & \answer{1/10} & \answer{3/10}
          \end{array}
        \right)
      \]
      Thus,
      \[
        A^{-1} = \left(
          \begin{array}{ccc}
            \answer{1/5} & \answer{-3/10} & \answer{3/5}  \\
            \answer{-1/5} & \answer{-1/5} & \answer{2/5}  \\
            \answer{1/10} & \answer{1/10} & \answer{3/10}
          \end{array}
        \right)
      \]
      We find
      \[
        \textbf{x} = \begin{pmatrix}
          \answer{2}+\answer{-3/5} + \answer{-12/5}\\
          \answer{-2}+\answer{-2/5} + \answer{-8/5}\\
          \answer{1}+\answer{1/5} + \answer{-6/5}
        \end{pmatrix}= \begin{pmatrix}
          \answer{-1}\\
          \answer{-4}\\
          \answer{0}
        \end{pmatrix}
      \]
      Thus, $x= \answer{-1}, y= \answer{-4}, z= \answer{0}$.
    \end{prompt}
  \end{enumerate}
\end{exercise}


\end{document} 
